\chapter{Plotting Support}\index{plotting}

\section{Plotting with {\normalfont \scshape Gnuplot}}

While \glc\ data can, of course, be plotted using whatever method you choose, two Perl modules are provided that we find useful for plotting \glc\ data. These are intended for use with \href{http://www.gnuplot.info/}{\normalfont \scshape GnuPlot} and with datasets stored as \href{http://pdl.perl.org/}{\normalfont \ttfamily PDL} variables. The first module, {\normalfont \ttfamily GnuPlot::PrettyPlots} plots lines and points with two color style (typically a lighter interior color and a darker border) with support for errorbars and limits (show as arrows) on points. The second, {\normalfont \ttfamily GnuPlot::LaTeX} provides a convenient way to process output from {\normalfont \scshape GnuPlot}'s {\normalfont \ttfamily epslatex} terminal into PDF files (suitable for inclusion in documents), PNG images with transparent backgrounds or \href{http://www.openoffice.org/}{OpenOffice} \href{http://www.wikimedia.org/wikipedia/en/wiki/OpenDocument}{ODG} files (suitable for inclusion into presentations\footnote{If you create an OpenOffice ODG file it's recommended that you covert it to a Metafile within OpenOffice before putting it into a presentation---this seems to prevent a bug which occasionally causes an element of the plot to be lost during saving\ldots}).

A typical use of these packages would look as follows:
\begin{verbatim}
use lib "./perl";
use PDL;
use GnuPlot::LaTeX;
use GnuPlot::PrettyPlots;

$outputFile = "myImage";
open($gnuPlot,"|gnuplot");
print $gnuPlot "set terminal epslatex color colortext lw 2 solid 7\n";
print $gnuPlot "set output '".$outputFile.".eps'\n";
print $gnuPlot "set xlabel '\$x-axis label\$'\n";
print $gnuPlot "set ylabel '\$y-axis label\$'\n";
print $gnuPlot "set lmargin screen 0.15\n";
print $gnuPlot "set rmargin screen 0.95\n";
print $gnuPlot "set bmargin screen 0.15\n";
print $gnuPlot "set tmargin screen 0.95\n";
print $gnuPlot "set key spacing 1.2\n";
print $gnuPlot "set key at screen 0.4,0.8\n";
print $gnuPlot "set key left bottom\n";
print $gnuPlot "set xrange [0.0:6.0]\n";
print $gnuPlot "set yrange [0.0:1.0]\n";
print $gnuPlot "set pointsize 2.0\n";
&PrettyPlots::Prepare_Dataset(\$plot,
			      $x1Data, $y1Data,
                              title => "First dataset",
                              style => line,
                              linePattern => 0,
                              weight => [7,3],
			      color => $PrettyPlots::colorPairs{'lightGoldenrod'}
                             );
&PrettyPlots::Prepare_Dataset(\$plot,
			      $x2Data, $y2Data,
			      errorDown => $errorDown,
			      errorUp   => $errorUp,
                              title => "Galacticus",
			      style => point,
                              symbol => [6,7],
                              weight => [5,3],
			      color => $PrettyPlots::colorPairs{'redYellow'}
                             );
&PrettyPlots::Plot_Datasets($gnuPlot,\$plot);
close($gnuPlot);
&LaTeX::GnuPlot2PNG($outputFile.".eps", backgroundColor => "#000080", margin => 1);
\end{verbatim}

The process begins by opening a pipe to {\normalfont \scshape GnuPlot} and specifying the {\normalfont \ttfamily epslatex} terminal along with {\normalfont \ttfamily color} and {\normalfont \ttfamily colortext} options, any line weight preferences and the output EPS file. This is followed by commands to set up the plot, including labels, ranges etc. Note that you \emph{must} specify margins manually\footnote{The {\normalfont \ttfamily GnuPlot::PrettyPlots} module works by generating multiple layers of plotting which are overlaid. Axes are only drawn for the first layer. If you do not specify margins manually, they will be computed automatically for each layer and so will not match up between all layers. This will result in data being plotted incorrectly.}. Following this are calls to {\normalfont \ttfamily \&PrettyPlots::Prepare\_Dataset} which prepares instructions for plotting of a single dataset. The first argument is a reference to a structure which will store the instructions, while the second and third arguments are PDLs containing the $x$ and $y$ data to be plotted. Following this are multiple options as follows:
\begin{description}
\item[{\normalfont \ttfamily title}] Gives the title of the dataset for inclusion in the plot key;
\item[{\normalfont \ttfamily style}] Specifies how the dataset should be drawn: either {\normalfont \ttfamily line}, {\normalfont \ttfamily point}, {\normalfont \ttfamily boxes}, or {\normalfont \ttfamily filledCurve};
\item{{\normalfont \ttfamily linePattern}} Specifies the line pattern (as defined for {\normalfont \scshape GnuPlot}'s {\normalfont \ttfamily lt} option) to use;
\item[{\normalfont \ttfamily symbol}] A two element list giving the symbol indices that should be used to plot the border and inner parts of each point respectively;
\item[{\normalfont \ttfamily weight}] A two element list giving the line weights to be used for border and inner parts of each point/line respectively;
\item[{\normalfont \ttfamily color}] A two element list giving the color of border and inner parts of each point/line respectively. Colors should be specified as {\normalfont \ttfamily \#RRGGBB} in hexadecimal. Several suitable color pairs and sequences of pairs are defined in the {\normalfont \ttfamily GnuPlot::PrettyPlots} module;
\item[{\normalfont \ttfamily pointSize}] Specifies the size of the points to be used;
\item[{\normalfont \ttfamily errorNNN}] Gives a PDL containing sizes of errors to be plotted on points in the up, down, left and right directions. A zero value will cause the error bar to be omitted, while a negative value will cause an arrow to be drawn with a length equal to the absolute value of the specified value;
\item[{\normalfont \ttfamily filledCurve}] If the {\normalfont \ttfamily filledCurve} style is used, this option specifies the type of filled curve ({\normalfont \ttfamily closed}, {\normalfont \ttfamily x1}, {\normalfont \ttfamily x2}, etc.---see the {\normalfont \scshape GnuPlot} {\normalfont \ttfamily help filledcurve} text for complete options). The default is {\normalfont \ttfamily closed};
\item[{\normalfont \ttfamily y2}] If the {\normalfont \ttfamily filledCurve} style is used along with the {\normalfont \ttfamily filledCurve}$=${\normalfont \ttfamily closed} option, this option is used to specify a second PDL of $y$-axis values. The region between this curve and the usual $y$-axis curve will be filled.
\end{description}
Once all datasets have been prepared, the call to {\normalfont \ttfamily \&PrettyPlots::Plot\_Datasets} will generate the EPS and \LaTeX\ files necessary to make the plot. This resulting plot can be converted to PDF, PNG or ODG form by calling {\normalfont \ttfamily \&LaTeX::GnuPlot2PDF}, {\normalfont \ttfamily \&LaTeX::GnuPlot2PNG} or {\normalfont \ttfamily \&LaTeX::GnuPlot2ODG} respectively. The EPS file will be replaced with the appropriate file. The {\normalfont \ttfamily \&LaTeX::GnuPlot2PNG} routine accepts an optional {\normalfont \ttfamily backgroundColor} argument in {\normalfont \ttfamily \#RRGGBB} format. If present, this color will be used to set the background color of the plot (otherwise white is assumed). Although the background is made transparent in the PNG, setting the background color is important as antialiasing will make use of this background. Note that both PNG and ODG options will switch black axes and labels to white\footnote{This is just a presonal preference for plots displayed in presentations---other options could be added}. Finally, the {\normalfont \ttfamily \&LaTeX::GnuPlot2PNG} routine accepts an optional {\normalfont \ttfamily margin} argument which specifies the size of the margin (in pixels) to be left around the plot when cropping.

The ODG option requires that both \href{http://www.cityinthesky.co.uk/opensource/pdf2svg}{{\normalfont \ttfamily pdf2svg}} and \href{http://www.haumacher.de/svg-import/}{{svg2office}} be installed on your system ({\normalfont \ttfamily svg2office} should be located in {\normalfont \ttfamily /usr/local/bin}).

\section{Merger Tree Diagrams with {\normalfont \scshape dot}}\index{merger trees!graphing}

The {\normalfont \scshape dot} command, which is a part of \href{http://www.graphviz.org/}{{\normalfont \scshape GraphViz}}\index{graphviz@{\normalfont \scshape GraphViz}} is useful for creating diagrams of merger trees. \glc\ provides a function to output the structure of any merger tree in {\normalfont \scshape GraphViz} format. This function, {\normalfont \ttfamily Merger\_Tree\_Dump}, is provided by the {\normalfont \ttfamily Merger\_Trees\_Dump} module. Usage is as follows:
\begin{lstlisting}[escapechar=@,breaklines,prebreak=\&,postbreak=\&]
call Merger_Tree_Dump(                                    &
     &                index                             , &
     &                baseNode                          , &
     &                highlightNodes     =highlightNodes, &
     &                backgroundColor    ='white'       , &
     &                nodeColor          ='black'       , &
     &                highlightColor     ='black'       , &
     &                edgeColor          ='#DDDDDD'     , &
     &                nodeStyle          ='solid'       , &
     &                highlightStyle     ='filled'      , &
     &                edgeStyle          ='solid'       , &
     &                labelNodes         =.false.       , &
     &                scaleNodesByLogMass=.true.        , &
     &                edgeLengthsToTimes =.true.        , &
     &                path               ='/my/path'      &
     &               )
\end{lstlisting}
Here {\normalfont \ttfamily index} is the tree index (successive calls to {\normalfont \ttfamily Merger\_Tree\_Dump} with the same index will result in a sequence of output files---see below), and {\normalfont \ttfamily baseNode} is a pointer to the base node of the tree to be dumped. All other arguments are optional:
\begin{description}
 \item [{\normalfont \ttfamily highlightNodes}] A list of node IDs. All nodes listed will be highlighted in the diagram;
 \item [{\normalfont \ttfamily backgroundColor}] The color for the background of the diagram;
 \item [{\normalfont \ttfamily nodeColor}] The color used to draw nodes;
 \item [{\normalfont \ttfamily highlightColor}] The color used for highlighted nodes;
 \item [{\normalfont \ttfamily edgeColor}] The color of edges (lines joining nodes);
 \item [{\normalfont \ttfamily nodeStyle}] The style to use when drawing nodes;
 \item [{\normalfont \ttfamily highlightStyle}] The style to use when drawing highlighted nodes;
 \item [{\normalfont \ttfamily edgeStyle}] The style to use when drawing edges;
 \item [{\normalfont \ttfamily labelNodes}] Specifies whether or not nodes should be labelled (labels consist of the node ID followed by the redshift);
 \item [{\normalfont \ttfamily scaleNodesByLogMass}] If true, the size of nodes will be set to be proportional to the logarithm of the node mass;
 \item [{\normalfont \ttfamily edgeLengthsToTimes}] If true, the spacing between parent and child nodes will be proportional to the logarithmic time interval between them.
 \item [{\normalfont \ttfamily path}] If present, write tree dumps into this directory. Otherwise, the current directory will be used.
\end{description}
All colors and styles are character strings and can be in any format understood by {\normalfont \scshape dot}. The tree structure will be dumped to file named {\normalfont \ttfamily mergerTreeDump:$\langle$ID$\rangle$:$\langle$N$\rangle$.gv} where {\normalfont \ttfamily $\langle$ID$\rangle$} is the index of the tree and {\normalfont \ttfamily $\langle$N$\rangle$} increasing incrementally from $1$ each time the same tree is consecutively dumped. These files can be processed using {\normalfont \scshape dot}. For example
\begin{verbatim}
 dot -Tps mergerTreeDump:1:1.gv > tree.ps
\end{verbatim}
will create a tree diagram as the PostScript file {\normalfont \ttfamily tree.ps}.
