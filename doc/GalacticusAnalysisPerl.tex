\documentclass[letterpaper,10pt,headsepline]{scrbook}
\usepackage[T1]{fontenc} 
\usepackage{natbib}
\usepackage{supertabular}
\usepackage{epsfig}
\usepackage{ifthen}
\usepackage{index}
\usepackage{xr-hyper}
\usepackage[backref,colorlinks]{hyperref}
\usepackage{listings}
\usepackage{verbatim}
\usepackage{hyphenat}
\usepackage{ragged2e}
\usepackage[acronym]{glossaries}
\usepackage{color}
\usepackage{tensor}
\usepackage{textcomp}

% Adaptive labelling.
\makeatletter
\newcommand{\iflabelexists}[3]{\@ifundefined{r@#1}{#3}{#2}}
\makeatother

% Names
\def\glc{{\normalfont \scshape Galacticus}}

% Reference main Galacticus documentation.
\externaldocument[main-]{../../galacticus/doc/Galacticus}[https://users.obs.carnegiescience.edu/abenson/galacticus/Galacticus_v0.9.4.pdf]

% Table of contents
\setcounter{tocdepth}{5}

% Margins.
\setlength{\topmargin}{0mm}
\setlength{\textwidth}{160mm}
\setlength{\textheight}{210mm}
\setlength{\oddsidemargin}{0mm}
\setlength{\evensidemargin}{0mm}

\makeglossary
\glstoctrue
\makeindex

\include{Glossary}

\begin{document}

\lstset{language=[95]Fortran}

\frontmatter

\pagestyle{empty}
\begin{center}
\includegraphics[width=125mm]{GalacticusLogo.png}\\

\includegraphics{New_Logo_Galaxy_192_Transparent.png}\\
A semi-analytic galaxy formation code.\\
Analyzing Galacticus Outputs Using Perl.\\

\copyright\ 2009, 2010 2011, 2012, 2013, 2014, 2015, 2016, 2017, 2018, Andrew Benson
\end{center}

\tableofcontents

\mainmatter
\pagestyle{headings}

\chapter{About Galacticus}

\glc\ is a semi-analytic model of galaxy formation. This document describes a collection of modules implemented in Perl designed for interacting with and analyzing the output of \glc\ models.

\subsection{License}

Copyright 2009, 2010, 2011, 2012, 2013, 2014, 2015, 2016, 2017, 2018, Andrew Benson \href{mailto:abenson@carnegiescience.edu}{\normalfont \ttfamily <abenson@carnegiescience.edu>}\\

\glc\ is free software: you can redistribute it and/or modify
it under the terms of the GNU General Public License as published by
the Free Software Foundation, either version 3 of the License, or
(at your option) any later version.

\glc\ is distributed in the hope that it will be useful,
but WITHOUT ANY WARRANTY; without even the implied warranty of
MERCHANTABILITY or FITNESS FOR A PARTICULAR PURPOSE.  See the
GNU General Public License for more details.

You should have received a copy of the GNU General Public License
along with \glc.  If not, see \href{http://www.gnu.org/licenses/}{\normalfont \ttfamily <http://www.gnu.org/licenses/>}.


\include{Analyze}

\include{Plotting}

\include{Tutorials}

\backmatter

\bibliographystyle{plainnat}
\bibliography{GalacticusAccented}

\printglossaries

\citeindextrue
\printindex

\end{document}
